\documentclass{article}
\usepackage[utf8]{inputenc}
\usepackage{graphicx,times,natbib,url,setspace,color,wrapfig,pdfpages,xcolor}
\usepackage[version=3]{mhchem}
\usepackage{siunitx}

%\usepackage[english]{babel}
\usepackage{amsmath,amssymb,mathabx}
\usepackage{parskip}
% Margins
\setlength{\topmargin}{-0.55in}
\setlength{\oddsidemargin}{-0.1in}
\setlength{\textwidth}{6.8in}
\setlength{\textheight}{9.0in}

%%%%%%%%%%%%%%%%%
\title{EESC GR6922: Atmospheric Radiation, Fall 2022 \\ Homework 2, October 15}
\author{Robert Pincus} 
\date{\today}

%%%%%%%%%%%%%%%%%
\begin{document}

\maketitle
%%%%%%%%%%%%%%%%%
While I get organized with the technology for doing problems in spectroscopy here are a pair of open-ended problems with aim of connecting the abstract ideas to real-world measurements. I'm allowing a bit less than a week; let me know if this is too ambitious. 

The main DEES office on the fifth floor of Schemmerhorn has a few infrared thermometers you can borrow, as well as some sheets of transparency material. You can check ahead by emailing Kaleigh Matthews (kaleighm@ldeo.columbia.edu) and/or  Julianna Russo (jr4432@columbia.edu). 

\section{Limb brightening}

Use the thermometer to measure the apparent temperature of the sky as a function of the polar angle, i.e. the angle from directly overhead to the horizon. If weather permits make measurements in both clear and cloudy skies. It might also be interesting to compare day and night. Plot the data and explain the signal. 

\section{Not-so-transparent transparency material} 
Transparency material is not actually so transparent in the infrared. Use the infrared thermometer to estimate the band-integrated transmissivity and apparent optical thickness of a single sheet of transparency material.  

\end{document}
